%!TeX root=main.tex

\section{Error absoluto y error relativo}

Cuando decimos que $\pi$ vale aproximadamente $3.14$,
que la distancia de Cádiz a Granada es de aproximadamente
trescientos kilómetros o que hemos pescado una dorada de
3 kilos cuando en realidad pesaba 1 kilo,
estamos cometiendo errores.
Existen dos maneras de contabilizar los errores, el
error absoluto que mide la diferencia que hay entre
la aproximación y el valor real,
y el error relativo, que mide la proporción de la diferencia
con el valor exacto.

Uno de los errores que aparecen en los cálculos numéricos
provienen de la manera que se guarda los valores en el ordenador.
Mostremos un caso donde se aprecie.

\input{errores-1}

Lo que ocurre es que cuando escribimos un número en decimal,
como es $0.1$, el ordenador aproxima al valor más cercano
en la representación en base 2 en un tipo de formato que
se llama coma flotante.

En este caso se puede calcular el error absoluto cometido
al aproximar $0.1$ de la siguiente manera.

\input{errores-2}
