%!TeX root=main.tex

\section{Números y precisión}

Debido a la naturaleza de los ordenadores y su capacidad
limitada, los cálculos numéricos no serán exactos.
Existirán ciertos errores que aparecerán debido a diferentes
factores.
Uno de los errores proviene de trabajar con una aproximación
en lugar del número exacto.
Por ejemplo, en numérico no se va a obtener el número $\sqrt{2}$.

\input{precision-1}

Está claro que no se podrá calcular numericamente de forma
exacta el número $\sqrt{2}$. Aunque maxima sí que tiene un
tipo de número que se conoce commo \maximain{bfloat} que
puede trabajar en precisión arbitraria.
Esto significa que se puede indicar el número de dígitos
que se va a trabajar.
El número de dígitos se controla con la variable
\maximain{fpprec} y para definir números de este tipo
se usa la función \maximain{bfloat}.

\input{precision-2}
