%!TeX root=main.tex

\section{Bucle \maximain{for}}

Un bucle es un elemento de programación para repetir
comandos de forma repetida.
Uno de los tipos de bucles es el \maximain{for},
este bucle se repite usando una variable que irá tomando
diferentes valores.
El comando más simple es
\begin{center}
	\maximain{for} nombrevar:valorinicial
	\maximain{thru} valorfinal 
	\maximain{do} expresión.
\end{center}

Veamos un ejemplo.

\begin{maximai}
	for i:1 thru 5 do print("i = ", i)$
\end{maximai}
\newcount\co
\co=1
\loop
\begin{maximaop}
	\mbox{ i = }\the\co
\end{maximaop}
\advance \co 1
\ifnum \co<6
\repeat

En el ejemplo anterior se puede ver que la variable
\maximain{i} va tomando los valores desde $1$ hasta $5$.

Si se quiere aumentar la variable del bucle \maximain{for}
en un valor diferente a la unidad se puede indicar con la
siguiente manera.
\begin{center}
	\maximain{for} nombrevar:valorinicial
	\maximain{step} paso
	\maximain{thru} valorfinal 
	\maximain{do} expresión.
\end{center}

\begin{maximai}
	for i:1 step 2 thru 10 do print("i = ", i)$
\end{maximai}
\co=1
\loop
\begin{maximaop}
	\mbox{ i = }\the\co
\end{maximaop}
\advance \co 2
\ifnum \co<11
\repeat

\begin{maximai}
	for i:10 step -1 thru 1 do print("i = ", i)$
\end{maximai}
\co=10
\loop
\begin{maximaop}
	\mbox{ i = }\the\co
\end{maximaop}
\advance \co -1
\ifnum \co>0
\repeat

Se puede calcular la suma de los cuadrados de los primeros
100 números usando \maximain{for} y una variable que lleve
la cuenta de la suma.

\begin{maximai}
suma:0$
\end{maximai}
\begin{maximal}
for i:1 thru 100 do suma:suma+i^2$
\end{maximal}
\begin{maximal}
print("La suma total es: ", suma)$
\end{maximal}
\begin{maximaop}
\mbox{ La suma total es: 338350 }
\end{maximaop}

Nos damos cuenta que para el cálculo de la suma de los cuadrados
de los 100 primeros números se hace uso de la variable auxiliar
\maximain{suma}, también se puede pensar que 100 es un parámetro
para el problema, así que se puede crear la función que devuelva
el valor de la suma de los cuadrados de los \maximain{n} primeros
números.

\begin{maximai}
sumacuadrados(n):=block([suma:0],
\end{maximai}
\begin{maximal}
for i:1 thru n do suma:suma+i^2,
\end{maximal}
\begin{maximal}
suma)$
\end{maximal}

\begin{maximai}
sumacuadrados(100);
\end{maximai}
\begin{maximao}
338350
\end{maximao}

\begin{maximai}
sumacuadrados(101);
\end{maximai}
\begin{maximao}
348551
\end{maximao}

\begin{maximai}
sumacuadrados(35);
\end{maximai}
\begin{maximao}
%c sumacuadrados(n):=block([suma:0],
%c for i:1 thru n do suma:suma+i^2,
%c suma)$
%c sumacuadrados(35)$
%d
14910
\end{maximao}

En el siguiente ejemplo usamos un bucle \maximain{for}
para el cálculo del seno de los ángulos desde 0 hasta
$2\pi$ con un paso de $\pi/4$.

\begin{maximai}
for k:0 thru 8 do (angulo:k*%pi/4,
\end{maximai}\begin{maximal}
print("El valor del seno de ", angulo, " es ", sin(angulo))
\end{maximal}\begin{maximal}
	)$
\end{maximal}

%c out=[]$
%c for k:0 thru 8 do (angulo:k*%pi/4,
%c out[k]:["El valor del seno de ", angulo, " es ", sin(angulo)])$
%e \begin{maximaop}
\begin{maximaop}
%c out[0][1]$
%d
\mbox{ El valor del seno de  }
%c out[0][2]$
%d
0
%c out[0][3]$
%d
\mbox{  es  }
%c out[0][4]$
%d
0
%e \end{maximaop}
\end{maximaop}
%e \begin{maximaop}
\begin{maximaop}
%c out[1][1]$
%d
\mbox{ El valor del seno de  }
%c out[1][2]$
%d
{{\pi}\over{4}}
%c out[1][3]$
%d
\mbox{  es  }
%c out[1][4]$
%d
{{1}\over{\sqrt{2}}}
%e \end{maximaop}
\end{maximaop}
%e \begin{maximaop}
\begin{maximaop}
%c out[2][1]$
%d
\mbox{ El valor del seno de  }
%c out[2][2]$
%d
{{\pi}\over{2}}
%c out[2][3]$
%d
\mbox{  es  }
%c out[2][4]$
%d
1
%e \end{maximaop}
\end{maximaop}
%e \begin{maximaop}
\begin{maximaop}
%c out[3][1]$
%d
\mbox{ El valor del seno de  }
%c out[3][2]$
%d
{{3\,\pi}\over{4}}
%c out[3][3]$
%d
\mbox{  es  }
%c out[3][4]$
%d
{{1}\over{\sqrt{2}}}
%e \end{maximaop}
\end{maximaop}
%e \begin{maximaop}
\begin{maximaop}
%c out[4][1]$
%d
\mbox{ El valor del seno de  }
%c out[4][2]$
%d
\pi
%c out[4][3]$
%d
\mbox{  es  }
%c out[4][4]$
%d
0
%e \end{maximaop}
\end{maximaop}
%e \begin{maximaop}
\begin{maximaop}
%c out[5][1]$
%d
\mbox{ El valor del seno de  }
%c out[5][2]$
%d
{{5\,\pi}\over{4}}
%c out[5][3]$
%d
\mbox{  es  }
%c out[5][4]$
%d
-{{1}\over{\sqrt{2}}}
%e \end{maximaop}
\end{maximaop}
%e \begin{maximaop}
\begin{maximaop}
%c out[6][1]$
%d
\mbox{ El valor del seno de  }
%c out[6][2]$
%d
{{3\,\pi}\over{2}}
%c out[6][3]$
%d
\mbox{  es  }
%c out[6][4]$
%d
-1
%e \end{maximaop}
\end{maximaop}
%e \begin{maximaop}
\begin{maximaop}
%c out[7][1]$
%d
\mbox{ El valor del seno de  }
%c out[7][2]$
%d
{{7\,\pi}\over{4}}
%c out[7][3]$
%d
\mbox{  es  }
%c out[7][4]$
%d
-{{1}\over{\sqrt{2}}}
%e \end{maximaop}
\end{maximaop}
%e \begin{maximaop}
\begin{maximaop}
%c out[8][1]$
%d
\mbox{ El valor del seno de  }
%c out[8][2]$
%d
2\,\pi
%c out[8][3]$
%d
\mbox{  es  }
%c out[8][4]$
%d
0
%e \end{maximaop}
\end{maximaop}

En el siguiente ejemplo se usa un bucle para crear una lista,
se empieza con la lista vacía y se van añadiendo elementos.

\begin{maximai}
lista:[]$
\end{maximai}\begin{maximal}
for i:1 thru 5 do
\end{maximal}\begin{maximal}
lista:append(lista,[i^2])$
\end{maximal}\begin{maximal}
lista;
\end{maximal}
\begin{maximao}
\left[ 1 , 4 , 9 , 16 , 25 \right] 
\end{maximao}

A continuación usamos un bucle para crear una lista de 25 elementos
en el que el elemento siguiente es $.5$ por el anterior, siendo el
primer elemento $2.6$.

\begin{maximai}
lista:[]$
\end{maximai}\begin{maximal}
for i:1 thru 25 do
\end{maximal}\begin{maximal}
if i=1 then lista:append(lista,[2.6])
\end{maximal}\begin{maximal}
else lista:append(lista,[ lista[i-1]*.5 ])$
\end{maximal}\begin{maximal}
lista;
\end{maximal}
\begin{maximao}
\left[ 2.6 , 1.3 , 0.65 , 0.325 , 0.1625 , 0.08125 , 0.040625 , 
	\ldots
 %0.0203125 , 0.01015625 , 0.005078125 , 0.0025390625 , 0.00126953125
  %, 6.34765625 \times 10^{-4} , 3.173828125 \times 10^{-4} , 
 %1.5869140625 \times 10^{-4} , 7.9345703125 \times 10^{-5} , 
 %3.96728515625 \times 10^{-5} , 1.983642578125 \times 10^{-5} , 
 %9.918212890625 \times 10^{-6} , 4.9591064453125 \times 10^{-6} , 
 %2.47955322265625 \times 10^{-6} , 1.239776611328125 \times 10^{-6}
  %, 6.198883056640625 \times 10^{-7}
	, 
 3.099441528320312 \times 10^{-7} , 1.549720764160156 \times 10^{-7}
  \right] 
\end{maximao}

En maxima el bucle \maximain{for} puede tener la siguiente forma,
para acceder a todos los elementos de un vector.
\begin{center}
	\maximain{for} var
	\maximain{in} vector
	\maximain{do} expr
\end{center}

El siguiente ejemplo cuenta el número de números positivos de una lista.
\begin{maximai}
lista:[-5,2,-5,-6,7,8,-9,-10,-2,3,5];
\end{maximai}
\begin{maximao}
\left[ -5 , 2 , -5 , -6 , 7 , 8 , -9 , -10 , -2 , 3 , 5 \right] 
\end{maximao}

\begin{maximai}
nneg:0$
\end{maximai}\begin{maximal}
for v in lista do if v<0 then nneg:nneg+1$
\end{maximal}\begin{maximal}
nneg;
\end{maximal}
\begin{maximao}
6
\end{maximao}

A partir de lo anterior se podría crear una función que se llame
\maximain{numneg},
que admita como argumento un vector y devuelva el número de elementos
negativos.

\begin{maximai}
numneg(L):=block([nneg:0],
\end{maximai}\begin{maximal}
for v in L do if v<0 then nneg:nneg+1,
\end{maximal}\begin{maximal}
nneg)$
\end{maximal}

\begin{maximai}
numneg([-1,-1,-1,1,1,-1]);
\end{maximai}
\begin{maximao}
4
\end{maximao}

A continuación se muestra un ejemplo en el que se repite la evaluación
de diferentes funciones.

\begin{maximai}
for f in [log, cos, sin] do print(f(1))$
\end{maximai}

%c out:[]$
%c for f in [log, cos, sin] do out:append(out,[f(1)])$
%e \begin{maximaop}
\begin{maximaop}
%c out[1]$
%d
0
%e \end{maximaop}
\end{maximaop}
%e \begin{maximaop}
\begin{maximaop}
%c out[2]$
%d
\cos 1
%e \end{maximaop}
\end{maximaop}
%e \begin{maximaop}
\begin{maximaop}
%c out[3]$
%d
\sin 1
%e \end{maximaop}
\end{maximaop}
