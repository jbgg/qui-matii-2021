%!TeX root=main.tex

\section{Lógica}

\subsection*{Valores lógicos}

En lógica se estudia la veracidad de las proposiciones,
esto es, si una proposición puede ser cierta o falsa.
En maxima existe los valores
\maximain{true} y \maximain{false}.
Aunque cuando se hace comprobaciones de variables que
no conoce ni sus valores ni ninguna propiedad puede que
resulte en \maximain{unknown}.
A continuación mostramos algunos ejemplos.

\begin{maximai}
is(2<3);
\end{maximai}
\begin{maximao}
\mathbf{true}
\end{maximao}

\begin{maximai}
is(3<2);
\end{maximai}
\begin{maximao}
\mathbf{false}
\end{maximao}

\begin{maximai}
is(a<b);
\end{maximai}
\begin{maximao}
{\it unknown}
\end{maximao}

El comando \maximain{is} da la información sobre la veracidad
de una expresión.

\subsection*{Operaciones de comparación}

Existen relaciones entre números,
por ejemplo igualdad, mayor igual, menor, \ldots,
estas relaciones pueden ser ciertas o falsas.
En maxima existen relaciones de comparación, a continuación
se muestran.
\begin{itemize}
	\item \maximain{=}. Igualdad.
	\item \maximain{\#}. Distinto.
	\item \maximain{<}. Menor.
	\item \maximain{<=}. Menor o igual.
	\item \maximain{>}. Mayor.
	\item \maximain{>=}. Mayor o igual.
\end{itemize}

Estas relaciones se pueden usar dentro del comando
\maximain{is}, como se ha visto en la sección anterior.

\subsection*{Operadores lógicos}

Existen operaciones entre los valores lógicos,
estos son operadores lógicos.
Algunos operadores lógicos se muestran a continuación.
\begin{itemize}
	\item \maximain{and}. Este operador es el ``y'' lógico.
		Devuelve \maximain{true} siempre que todos los operandos
		son ciertos, en otro caso es \maximain{false}.
	\item \maximain{or}. Este operador es el ``o'' lógico.
		Devuelve \maximain{true} siempre que exista al menos
		un operando que sea cierto, en otro caso es \maximain{false}.
	\item \maximain{not}. Este es el operador negación, y solamente
		acepta un operando, devolviendo el valor opuesto.
\end{itemize}

Las operaciones anteriores sirven para comprobar diferentes situaciones.
Por ejemplo para probar que la variable \maximain{a} está en el intervalo
$[0,1]$ se puede hacer lo siguiente.

\begin{maximai}
is(0<a and a<1);
\end{maximai}
\begin{maximao}
{\it unknown}
\end{maximao}

El caso anterior el resultado es \textit{unknown} ya que la variable
\maximain{a} no tiene ningún valor.

Además de valores numéricos se pueden comparar expresiones,
recuerde que en maxima se trabaja por defecto con expresiones simbólicas.
Dos expresiones pueden resultar falsas comprobadas de forma simbólicas
aunque algebraicamente sean iguales, esto se debe a que la igualdad
de expresiones en maxima se verifica cuando está escrita de la misma manera.
Obsevese el siguiente ejemplo.

\begin{maximai}
is((x+1)^2=x^2+2*x+1);
\end{maximai}
\begin{maximao}
\mathbf{false}
\end{maximao}

\begin{maximai}
is(expand((x+1)^2)=x^2+2*x+1);
\end{maximai}
\begin{maximao}
\mathbf{true}
\end{maximao}

\subsection*{Suposiciones}

En ciertas ocasiones es necesario suponer que
una variable verifica alguna condición.
Los comandos \maximain{assume} y \maximain{forget}
sirven para ello.
Veamos cómo funciona en el siguiente caso.
Por una parte sabemos lo siguiente.

\begin{maximai}
is((x+1)^2>=0);
\end{maximai}
\begin{maximao}
\mathbf{true}
\end{maximao}

Pero si $x\neq-1$ entonces debería ser cierto que
\maximain{(x+1)\^{}2>0}.
Para que maxima tenga en cuenta que la variable \maximain{x}
verifica que es distinto a $-1$ y comprobar la desigualdad
anterior se procede como sigue.

\begin{maximai}
assume(notequal(x,-1));
\end{maximai}
\begin{maximao}
\left[ {\it notequal}\left(x , -1\right) \right] 
\end{maximao}

\begin{maximai}
is((x+1)^2>0);
\end{maximai}
\begin{maximao}
\mathbf{true}
\end{maximao}

A partir de ahora maxima ya sabe que $\maximain{x}\neq-1$.
Para que deje de suponer la condición anterior habrá que usar
\maximain{forget}.

\begin{maximai}
forget(notequal(x,-1));
\end{maximai}
\begin{maximao}
\left[ {\it notequal}\left(x , -1\right) \right] 
\end{maximao}

A partir de ahora se tiene lo siguiente.

\begin{maximai}
is((x+1)^2>0);
\end{maximai}
\begin{maximao}
{\it unknown}
\end{maximao}
