%!TeX root=main.tex

\section{Funciones}

Maxima ya incorpora algunas funciones ya definidas,
pero también se pueden incorporar funciones definidas
por el usuario.
Cuando nos referimos a funciones en maxima no solamente
nos referimos a las funciones matemáticas con valores numéricos,
sino a funciones que pueden usar todo tipo de datos de maxima.
Un ejemplo de funciones que no son matemáticas son las funciones
que se aplican a las listas.

Las funciones sirven para automatizar un proceso con datos sin determinar,
estos son los argumentos, que a la hora de usar la función toman los valores
que se le indique.

Por ejemplo, podríamos definir el logaritmo en base 2 como mostramos.
\begin{maximai}
	log2(x):=log(x)/log(2)$
\end{maximai}

Y usar la función de la siguiente forma.
\begin{maximai}
	log2(32),numer;
\end{maximai}\begin{maximao}
	5.0
\end{maximao}

Y también se podría definir el logaritmo en base 3, 4, 5, \ldots
pero entonces nos damos cuenta que la base también puede ser un argumento
de la función, y así tendríamos todas las posibilidades.

\begin{maximai}
	logb(x,b):=log(x)/log(b)$
\end{maximai}\begin{maximai}
	logb(25,5),numer;
\end{maximai}\begin{maximao}
	2.0
\end{maximao}

A veces hace falta definir funciones que hacen algunos cálculos
intermedios, en este caso hay que definir la función con un comando
conocido como \maximain{block}.


\begin{maximai}
	f1(a,b,c):=block([u,v], u:b-c, v:a-b, u/v)$
\end{maximai}
\begin{maximai}
	f1(2,3,1);
\end{maximai}\begin{maximao}
	-2
\end{maximao}

En el ejemplo anterior se han usado las variables \maximain{u} y
\maximain{v} para guardar los cálculos intermedios, y el resultado
que se calcula en la función es el último, \maximain{u/v}.

\begin{maximai}
	f2(y):=block([u:y^2], u:u+3, u:u^2, u)$
\end{maximai}
\begin{maximai}
	f2(4);
\end{maximai}\begin{maximao}
	361
\end{maximao}

En el ejemplo anterior la variable auxiliar \maximain{u} se inicializa
con una expresión del argumento.

En ciertas ocasiones es útil escribir un mensaje por pantalla en cierto
momento en el cálulo de la función.


\begin{maximai}
	f3(x):=block(print("saludos"),x^2+1)$
\end{maximai}\begin{maximai}
	f3(-1);
\end{maximai}\begin{maximaop}
	\text{saludos}
\end{maximaop}\begin{maximao}
	2
\end{maximao}

En el siguiente ejemplo mostramos como se repite la llamada de la función.

\begin{maximai}
	create_list(f3(i^2),i,1,3);
\end{maximai}\begin{maximaop}
	\text{saludos}
\end{maximaop}\begin{maximaop}
	\text{saludos}
\end{maximaop}\begin{maximaop}
	\text{saludos}
\end{maximaop}\begin{maximao}
	\left[ 2 , 17 , 82 \right]
\end{maximao}

A continuación se muestra cómo se crearía una función que calcula
el módulo de un número imaginario.

\begin{maximai}
	f4(x,y):=block([re,im,g:x+%i*y], re:realpart(g), im:imagpart(g),
\end{maximai}\begin{maximal}
	sqrt(re^2+im^2))$
\end{maximal}\begin{maximai}
	f4(2,-1);
\end{maximai}\begin{maximao}
	\sqrt{5}
\end{maximao}

Existe una forma para admitir un número indeterminado de argumentos.

\begin{maximai}
	sumdif([x]):=x^2 . create_list( (-1)^(k+1),k,1,length(x))$
\end{maximai}\begin{maximai}
	sumdif(a,b,c,d,e,f,g);
\end{maximai}\begin{maximao}
	g^2-f^2+e^2-d^2+c^2-b^2+a^2
\end{maximao}\begin{maximai}
	sumdif(1,1,-1,0,4,5,0,-8,9,-1);
\end{maximai}\begin{maximao}
	8
\end{maximao}\begin{maximai}
	sumdif(2,u+v);
\end{maximai}\begin{maximao}
	4-\left(v+u\right)^2
\end{maximao}

Una función se puede pasar como argumento al definir una función.

\begin{maximai}
	funfun(G,a):=G(a^2)^2;
\end{maximai}\begin{maximai}
	funfun(log,2);
\end{maximai}\begin{maximao}
	\left(\log 4\right)^2
\end{maximao}

\begin{maximai}
	funfun2(G,[x]):=G(x)$
\end{maximai}\begin{maximai}
	funfun2(sin,%pi,%pi/2);
\end{maximai}\begin{maximao}
	\left[ 0 , 1 \right]
\end{maximao}

Un argumento de la función puede también ser una lista,
a continuación mostramos un ejemplo de ello.

\begin{maximai}
	ultimodelalista(L):=L[length(L)];
\end{maximai}\begin{maximai}
	ultimodelalista([1,-1,3,22,35,7]);
\end{maximai}\begin{maximao}
	7
\end{maximao}

\subsection*{Ejercicios}

\begin{enumerate}
		
	\item % raíz n-ésima
		Cree una función \maximain{raizn($a$,$n$)} que calcule la raíz
		\maximain{$n$}-ésima del número \maximain{$a$}.

	\item % derivada n-ésima de la función seno en un punto x
		Cree una función \maximain{dsen($x$,$n$)} que calcule la derivada
		\maximain{$n$}-ésima de la función $\sin$ en el punto \maximain{$x$}.


	\item % crear una función mayoramenor que calcule una lista
		% de mayor a menor... argumento...
		Cree una función que tenga como argumento una lista y
		calcule la lista ordenada de mayor a menor.

	\item % el ejercicio 1 de la práctica de listas creaba
		% una lista de puntos que formaban un hexágono,
		% generalizar ...
		% creando una función que admita, n, como argumento
		% y genere una lista con los puntos formando un
		% polígono regular de n puntos.
		% el nombre de la función será poligonoregular(n)
		% probarlo con el siguiente comando
		% wxplot2d([discrete, poligonoregular(n)], [style,points]);
		En el ejercicio 1 de la práctica sobre listas se calculaba
		una lista de puntos que eran los vértices de un hexágono regular.
		En este ejercicio se pide generalizarlo creando una función
		\maximain{poligonoregular($n$)} que devuelva una lista de puntos
		que sean los vértices de un polígono regular de \maximain{$n$} vértices.
		Compruebese la solución graficando los puntos.

	%\item % la función max(a,b,...) devuelve el máximo de los
		% valores. Pero no funciona con listas es decir
		% max([3,2]) no devuelve el máximo de la lista
		% crear la función maxlist(l) que admite una lista
		%La función \maximain{max($a$,$b$,\ldots)} está definida por maxima

\end{enumerate}
