%!TeX root=main.tex

\section{Bucle \maximain{while}}

El bucle \maximain{while} tiene la siguiente forma
\begin{center}
	\maximain{while} condición \maximain{do} expresión
\end{center}
donde se indica una condición y una expresión,
y dicha expresión se seguirá ejecutando mientras la
condición sea cierta. Cuando se quiera ejecutar más
de una orden se usará paréntesis separando por comas
las diferentes órdenes.

Veamos un ejemplo donde se usa una variable \maximain{n}
para la condición del bucle, en particular el bucle se ejecutará
siempre \maximain{n<5}. En este caso la variable \maximain{n}
se inicializa con el valor \maximain{0} y en cada iteración
se aumenta en una unidad.

Veamos el mismo ejemplo donde el valor de la variable
\maximain{n} se actualiza al principio de la ejecución.

\input{while-1.tex}

Mostramos ahora un ejemplo usando aleatoriedad.

\input{while-2.tex}

Observe que cada vez que se ejecute tendrá una
salida diferente, esto se debe a que la función
\maximain{random} dará un valor diferente.

\input{while-3.tex}

La condición del bucle se comprueba incluso antes de ejecutar
el bucle, así que puede ocurrir que la expresión del bucle no
se ejecute ninguna vez.

\input{while-4.tex}

Una funcionalidad importante de los bucles \maximain{while}
es que pueden parar de ejecutar en función de alguna condición.
A continuación vamos a encontrar el primer elemento no nulo
de una lista usando este tipo de bucle.

\input{while-5.tex}

El ejemplo anterior se puede realizar de la siguiente manera
donde se comprueba si el elemento de la lista sigue siendo
nulo en la propia condición. Se ha aprovechado el uso de paréntesis
en una condición en la que se ejecuta comandos en la condición,
la comprobación se realiza al último elemento del paréntesis.

\input{while-6.tex}

Se puede escribir una función que se llame \maximain{buscanonulo}
y que admita un lista, y devuelva el índice y el valor del
primer elemento no nulo del lista.

\input{while-7.tex}

Vamos a presentar a continuación el código que en una lista
de elementos que son lista de números encuentra la primera
que su suma es cero.
Primero definamos una lista para probar el código, llamemos a la lista
\maximain{ll}.
Y a continuación mostramos los comandos para el cálculo.

\input{while-8.tex}

Como de costumbre creemos una función que realice esta comprobación.

\input{while-9.tex}
