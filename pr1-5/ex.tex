%!TeX root=main.tex

\section*{Ejercicios}

\begin{enumerate}

	\item Escriba un comando usando el bucle \maximain{while}
para calcular la suma de los números impares entre 7 y 25, ambos inclusives.

	\item Escriba un comando usando el bucle \maximain{while}
para calcular la suma de los números positivos del lista.
Por ejemplo la suma de los números positivos de la lista
%ei lista:[2,3,-1,0,5];
\begin{maximai}
lista:[2,3,-1,0,5];
\end{maximai}
%do
\begin{maximao}
\left[ 2 , 3 , -1 , 0 , 5 \right] 
\end{maximao}
%e es
es
%c suma:0$ for v in lista do if v>0 then suma:suma+v$ suma;
%d $ $.
$
10
$.
Escriba una función llamada \maximain{sumapos} que haga el
cálculo anterior para una lista de números.

	\item Escriba un comando usando el bucle \maximain{while}
que verifique si el número 42 está en una lista o no.
Defina una función llamada \maximain{estaono(lista,n)}, siendo
el primer argumento una lista de números y el segundo argumento
un valor \maximain{n}, la función devolverá \maximain{true}
si el número \maximain{n} está en la lista \maximain{lista}.
Por ejemplo
%c estaono(lista,n):=block([esta:false,i:1],
%c while i<=length(lista) and not esta do
%c (if lista[i]=n then esta:true, i:i+1),
%c esta)$
%ei estaono([7,-4,8,9,23,42,0],42);
\begin{maximai}
estaono([7,-4,8,9,23,42,0],42);
\end{maximai}
%do
\begin{maximao}
\mathbf{true}
\end{maximao}
%ei estaono([7,-4,8,9,23,42,0],142);
\begin{maximai}
estaono([7,-4,8,9,23,42,0],142);
\end{maximai}
%do
\begin{maximao}
\mathbf{false}
\end{maximao}

\end{enumerate}
