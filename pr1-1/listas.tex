%!TeX root=main.tex

\section{Listas y tablas}

\subsection*{Creando listas}

Las listas son objetos que almacenan valores de forma ordenada.
A continuación mostramos cómo se pueden crear listas a partir de
los valores que la forman.

\begin{maximai}
	lista:[2,-3,5,6];
\end{maximai}\begin{maximao}
	\left[ 2 , -3 , 5 , 6 \right]
\end{maximao}\begin{maximai}
	otralista:[1,2+1/2,[a,2,c],1+3*x+y^2];
\end{maximai}\begin{maximao}
	\left[ 1 , {{5}\over{2}} , \left[ a , 2 , c \right]  , y^2+3\,x+1
	\right]
\end{maximao}

Una lista puede no tener ningún valor, esta es la lista vacía.
\begin{maximai}
	listavacia:[];
\end{maximai}\begin{maximao}
	\left[\  \right]
\end{maximao}

Existe un tipo de lista especial, que llamamos tabla, este tipo de lista
almacena en cada entrada una nueva lista con la particularidad que todas
estas listas tienen el mismo número de elementos.

Por ejemplo una tabla almacenando dos valores cada vez.
\begin{maximai}
	tabla:[[1,2],[-3,4],[8,7],[-10,11]];
\end{maximai}\begin{maximao}
	\left[ \left[ 1 , 2 \right]  , \left[ -3 , 4 \right]  , \left[ 8 ,
	7 \right]  , \left[ -10 , 11 \right]  \right]
\end{maximao}
Y otro ejemplo de una tabla almacenando tres valores.
\begin{maximai}
	otratabla:[[1,2,2],[-3,8,4],[8,-2,7],[-10,99,11]];
\end{maximai}\begin{maximao}
	\left[ \left[ 1 , 2 , 2 \right]  , \left[ -3 , 8 , 4 \right]  ,
	\left[ 8 , -2 , 7 \right]  , \left[ -10 , 99 , 11 \right]  \right]
\end{maximao}

Anteriormente se han creado las listas expresando explícitamente todos
sus elementos, pero también hay posibilidades de crear listas a partir
de una expresión, esto se puede hacer con el comando
\maximain{create\_list}, su usa de la siguiente manera
\begin{center}
	\maximain{create\_list(expr, i, $\texttt{i}_0$, $\texttt{i}_1$)}
\end{center}
y crea la lista evaluando en expresión el valor de \maximain{i},
desde \maximain{$\texttt{i}_0$} hasta \maximain{$\texttt{i}_0$}.

\begin{maximai}
	create_list(2*i,i,1,10);
\end{maximai}\begin{maximao}
	\left[ 2 , 4 , 6 , 8 , 10 , 12 , 14 , 16 , 18 , 20 \right]
\end{maximao}

\begin{maximai}
	create_list([i, 2*i],i,1,10);
\end{maximai}\begin{maximao}
	\left[ \left[ 1 , 2 \right]  , \left[ 2 , 4 \right]  , \left[ 3 , 6
	\right]  , \left[ 4 , 8 \right]  , \left[ 5 , 10 \right]  , \left[
		6 , 12 \right]  , \left[ 7 , 14 \right]  , \left[ 8 , 16 \right]  ,
	\left[ 9 , 18 \right]  , \left[ 10 , 20 \right]  \right]
\end{maximao}

Otra manera de usar la función
\maximain{create\_list} es la siguiente
\begin{center}
	\maximain{create\_list(expr, i, lista)}
\end{center}
evaluando los valores de \maximain{i} en cada
valor de la lista \maximain{lista}.

\begin{maximai}
	xx:create_list(%pi*i/8, i, 1, 16);
\end{maximai}\begin{maximao}
	\left[ {{\pi}\over{8}} , {{\pi}\over{4}} , {{3\,\pi}\over{8}} , {{
		\pi}\over{2}} , {{5\,\pi}\over{8}} , {{3\,\pi}\over{4}} , {{7\,\pi
	}\over{8}} , \pi , {{9\,\pi}\over{8}} , {{5\,\pi}\over{4}} , {{11\,
	\pi}\over{8}} , {{3\,\pi}\over{2}} , {{13\,\pi}\over{8}} , {{7\,\pi
	}\over{4}} , {{15\,\pi}\over{8}} , 2\,\pi \right]
\end{maximao}\begin{maximai}
	puntos:create_list([x,sin(x)], x, xx);
\end{maximai}\begin{maximao}
	\left[ \left[ {{\pi}\over{8}} , \sin \left({{\pi}\over{8}}\right)
	\right]  , \left[ {{\pi}\over{4}} , {{1}\over{\sqrt{2}}} \right]
	, \left[ {{3\,\pi}\over{8}} , \sin \left({{3\,\pi}\over{8}}\right)
	\right]  ,
	%\left[ {{\pi}\over{2}} , 1 \right]  , \left[ {{5\,\pi
	%\over{8}} , \sin \left({{5\,\pi}\over{8}}\right) \right]  , \left[
	%{{3\,\pi}\over{4}} , {{1}\over{\sqrt{2}}} \right]  , \left[ {{7\,\pi
	%}\over{8}} , \sin \left({{7\,\pi}\over{8}}\right) \right]  , \left[
	%\pi , 0 \right]  , \left[ {{9\,\pi}\over{8}} , \sin \left({{9\,\pi
	%}\over{8}}\right) \right]  , \left[ {{5\,\pi}\over{4}} , -{{1}\over{
	%\sqrt{2}}} \right]  , \left[ {{11\,\pi}\over{8}} , \sin \left({{11\,
	%\pi}\over{8}}\right) \right]  , \left[ {{3\,\pi}\over{2}} , -1
	%\right]  , \left[ {{13\,\pi}\over{8}} , \sin \left({{13\,\pi}\over{
	%8}}\right) \right]  , \left[ {{7\,\pi}\over{4}} , -{{1}\over{\sqrt{2
	%}}} \right]
	\ldots
	, \left[ {{15\,\pi}\over{8}} , \sin \left({{15\,\pi
	}\over{8}}\right) \right]  , \left[ 2\,\pi , 0 \right]  \right]
\end{maximao}

Para obtener un elemento de una lista debemos indicar su posición
entre corchetes. Por ejemplo el primer elemento de \maximain{xx}.
\begin{maximai}
	xx[1];
\end{maximai}\begin{maximao}
	{{\pi}\over{8}}
\end{maximao}

O el elemento tercero de la tabla \maximain{puntos}.
\begin{maximai}
	puntos[3];
\end{maximai}\begin{maximao}
	\left[ {{3\,\pi}\over{8}} , \sin \left({{3\,\pi}\over{8}}\right)
	\right]
\end{maximao}

Este último a su vez es una lista, así que podríamos obtener
el segundo elemento del tercer elemento de la tabla \maximain{puntos}
de la siguiente manera.
\begin{maximai}
	puntos[3][2];
\end{maximai}\begin{maximao}
	\sin \left({{3\,\pi}\over{8}}\right)
\end{maximao}

El tamaño de una lista se puede obtener con la función
\maximain{length}.
\begin{maximai}
	length(xx);
\end{maximai}\begin{maximao}
	16
\end{maximao}\begin{maximai}
	length(puntos);
\end{maximai}\begin{maximao}
	16
\end{maximao}\begin{maximai}
	length(puntos[1]);
\end{maximai}\begin{maximao}
	2
\end{maximao}\begin{maximai}
	length([]);
\end{maximai}\begin{maximao}
	0
\end{maximao}

\subsection*{Representación de tablas}

Los puntos anteriores se podrían representar de la siguiente manera.

\begin{maximai}
	wxplot2d([discrete, puntos], [style,points]);
\end{maximai}\begin{maximat}
	\begin{center}
		\includegraphics[scale=.5]{wxplot2d_puntos.pdf}
	\end{center}
\end{maximat}

En el caso anterior se ha evaluado una función que ya
tiene definida maxima como \maximain{cos}.
Pero también se puede usar una función definida por el usuario.
\begin{maximai}
	f(x):=x^3-3*x^2+5*x-2$
\end{maximai}\begin{maximai}
	xx2:create_list(2*i+1, i, 0, 4);
\end{maximai}\begin{maximao}
	\left[ 1 , 3 , 5 , 7 , 9 \right]
\end{maximao}\begin{maximai}
	puntos2:create_list(f(x),x,xx2);
\end{maximai}\begin{maximao}
	\left[ 1 , 13 , 73 , 229 , 529 \right] 
\end{maximao}\begin{maximai}
	wxplot2d([discrete, puntos2], [style,points]);
\end{maximai}\begin{maximat}
	\begin{center}
		\includegraphics[scale=.5]{wxplot2d_puntos2.pdf}
	\end{center}
\end{maximat}

En el ejemplo anterior se ha representado los puntos
de la función $f$ en los que los valores
de abscisas son los números impares del 1 al 9.


\subsection*{Operaciones sobre listas}

A partir de listas se pueden calcular nuevas listas.
Una posibilidad es aplicar operaciones artiméticas
entre listas.

\begin{maximai}
	a:[1,2,3,4,5]$
\end{maximai}\begin{maximal}
	b:[6,7,8,9,0]$
\end{maximal}
\begin{maximai}
	a+b;
\end{maximai}\begin{maximao}
	\left[ 7 , 9 , 11 , 13 , 5 \right]
\end{maximao}\begin{maximai}
	a-b;
\end{maximai}\begin{maximao}
	\left[ -5 , -5 , -5 , -5 , 5 \right]
\end{maximao}\begin{maximai}
	a*b;
\end{maximai}\begin{maximao}
	\left[ 6 , 14 , 24 , 36 , 0 \right]
\end{maximao}\begin{maximai}
	b/a;
\end{maximai}\begin{maximao}
	\left[ 6 , {{7}\over{2}} , {{8}\over{3}} , {{9}\over{4}} , 0
	\right]
\end{maximao}\begin{maximai}
	a^2;
\end{maximai}\begin{maximao}
	\left[ 1 , 4 , 9 , 16 , 25 \right]
\end{maximao}\begin{maximai}
	sqrt(b);
\end{maximai}\begin{maximao}
	\left[ \sqrt{6} , \sqrt{7} , 2^{{{3}\over{2}}} , 3 , 0 \right]
\end{maximao}

Como vemos las operaciones se aplica elemento a elemento.

El producto escalar se calcula con el operador punto (\maximain{.}).
\begin{maximai}
	a.b;
\end{maximai}\begin{maximao}
	80
\end{maximao}

Existen otras operaciones definidas en maxima como las que veremos
a continuación, pero para ello definamos previamente una lista para
y una tabla.

\begin{maximai}
	lista1:[2,-3,5,6];
\end{maximai}\begin{maximao}
	\left[ 2 , -3 , 5 , 6 \right]
\end{maximao}\begin{maximai}
	tabla:[[1,2],[-3,4],[8,7],[-10,11]];
\end{maximai}\begin{maximao}
	\left[ \left[ 1 , 2 \right]  , \left[ -3 , 4 \right]  , \left[ 8 ,
	7 \right]  , \left[ -10 , 11 \right]  \right]
\end{maximao}

\begin{itemize}

	\item \maximain{delete}. Devuelve una nueva lista con los valores
		eliminados.
		\begin{maximai}
			lista2:delete(-3,lista1);
		\end{maximai}\begin{maximao}
			\left[ 2 , 5 , 6 \right]
		\end{maximao}

	\item \maximain{cons}. Devuelve una nueva lista añadiendo al principio
		de la lista el valor que se indique.
		\begin{maximai}
			lista3:cons(-5,lista1);
		\end{maximai}\begin{maximao}
			\left[ -5 , 2 , -3 , 5 , 6 \right]
		\end{maximao}

	\item \maximain{append}. Calcula una nueva lista uniendo la primera
		lista con la segunda.
		\begin{maximai}
			lista4:append(lista2,lista3);
		\end{maximai}\begin{maximao}
			\left[ 2 , 5 , 6 , -5 , 2 , -3 , 5 , 6 \right]
		\end{maximao}

	\item \maximain{reverse}. Calcula una nueva lista cambiando los
		valores de orden.
		\begin{maximai}
			lista5:reverse(lista4);
		\end{maximai}\begin{maximao}
			\left[ 6 , 5 , -3 , 2 , -5 , 6 , 5 , 2 \right]
		\end{maximao}

	\item \maximain{unique}. Devuelve una nueva lista eliminando los
		valores repetidos.
		\begin{maximai}
			lista6:unique(lista5);
		\end{maximai}\begin{maximao}
			\left[ -5 , -3 , 2 , 5 , 6 \right]
		\end{maximao}

	\item \maximain{sort}. Calcula una nueva lista ordenando los valores
		de menor a mayor.
		\begin{maximai}
			lista7:sort(lista6);
		\end{maximai}\begin{maximao}
			\left[ -5 , -3 , 2 , 5 , 6 \right]
		\end{maximao}

	\item \maximain{flatten}. Esta función aplicada a una tabla unifica
		los valores en una única lista.
		\begin{maximai}
			flatten(tabla);
		\end{maximai}\begin{maximao}
			\left[ 1 , 2 , -3 , 4 , 8 , 7 , -10 , 11 \right]
		\end{maximao}

\end{itemize}

\subsection*{Ejercicios}

\begin{enumerate}

	\item \begin{enumerate}
			\item Definir la función $f(k)=\frac{2k\pi}{6}$.
			\item Definir una lista evaluando la función en los valores
				$k=0,\ldots,5$.
			\item Crear una lista de puntos $(\cos x, \sin x)$, para
				cada valor de $x$ de la lista anterior.
			\item Representar gráficamente los puntos anteriores.
	\end{enumerate}

\item \begin{enumerate}
		\item Definir la función
			\begin{equation*}
				f(x) = \left( \frac{\sqrt{2}\cos x}{\sin^2 x +1},
				\frac{\sqrt{2}\cos x\sin x}{\sin^2 x +1},\right).
			\end{equation*}
		\item Crear una lista de valores desde 0 hasta 10,
			con un espacio entre valores de $0.1$
		\item Crear una tabla con las imágenes de $f$ en los
			valores anteriores.
		\item Representar gráficamente la tabla anterior.
\end{enumerate}

		\item \begin{enumerate}
				\item Crear una tabla de 5 elementos con los puntos
					$(n,2n^3,n-2)$ empezando en $n=0$.
				\item Obtener el elemento cuarto de la tabla anterior.
				\item Obtener el elemento segundo del tercer elemento
					de la tabla.
				\item Escribir la tabla como una única lista.
				\item Ordenar la lista anterior.
				\item Ordenar la lista de mayor a menor.
		\end{enumerate}

\end{enumerate}
